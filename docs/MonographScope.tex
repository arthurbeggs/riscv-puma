\documentclass{article}
\usepackage[utf8]{inputenc}

\title{Monografia Arthur}
\author{Arthur de Matos Beggs e Marcus Vinicius Lamar}
\date{March 2017}

\begin{document}

\maketitle

\break

\section{Introdução}

    {O mercado de trabalho está a cada dia mais exigente, sempre buscando profissionais que conheçam as melhores e mais recentes ferramentas disponíveis. Além disso, muitos universitários se sentem desestimulados ao estudarem assuntos desatualizados e com baixa possibilidade de aproveitamento do conteúdo no mercado de trabalho. Isso alimenta o desinteresse pelos temas abordados e, em muitos casos, leva à evasão escolar. Assim, é importante renovar as matérias com novas tecnologias e tendências de mercado sempre que possível a fim de instigar o interesse dos discentes e formar profissionais mais capacitados e preparados para as demandas da atualidade.}

    {Hoje, a disciplina OAC (Organização e Arquitetura de Computadores) é ministrada utilizando a arquitetura \textit{MIPS32} (\textit{Microprocessor without Interlocked Pipeline Stages}). Apesar da arquitetura \textit{MIPS32} ainda ter grande força no meio acadêmico (em boa parte devido a sua simplicidade e extensa bibliografia), sua aplicação na indústria tem diminuído consideravelmente na última década.}

    {Embora a curva de aprendizagem de linguagens \textit{Assembly} de alguns processadores \textit{RISC} (\textit{Reduced Instruction Set Computing}) seja relativamente baixa para quem já conhece o \textit{Assembly MIPS32}, aprender uma arquitetura atual traz o benefício de conhecer o \textit{estado da arte} da organização e arquitetura de computadores.}

    {Para a modernização da disciplina, foi escolhida a \textit{ISA} (\textit{Instruction Set Architecture}) \textit{RISC-V} desenvolvida na Divisão de Ciência da Computação da Universidade da Califórnia, Berkeley.}


\section{Por que \textit{RISC-V?}}

    {A \textit{ISA RISC-V} (lê-se \textit{"risk-five"}) é uma arquitetura \textit{open source} com licença \textit{BSD} (\textit{Berkeley Software Distribution}), o que permite o seu livre uso para quaisquer fins, sem distinção de se o trabalho possui código aberto ou fechado. Tal característica possibilita que grandes fabricantes utilizem a arquitetura para criar a base de seus produtos, mas que o desenvolvimento de qualquer subconjunto de instruções não-\textit{standard} ou seus métodos de implementação possam ser fechados, protegendo a propriedade intelectual e estimulando investimentos em pesquisa e desenvolvimento.}

    {Empresas como Google, IBM, AMD, Nvidia, Hewlett Packard, Microsoft, Oracle, Qualcomm e Western Digital são algumas das fundadoras e investidoras da \textit{RISC-V Foundation}, órgão responsável pela governança da arquitetura. Isso demonstra o interesse das gigantes do mercado no sucesso e disseminação da arquitetura.}

    {O conjunto de instruções foi desenvolvido tendo em mente seu uso em diversas escalas: sistemas embarcados, \textit{smartphones}, computadores pessoais, servidores e supercomputadores o que permitirá maior reuso de \textit{software} e maior integração de \textit{hardware}.}

    {Outro fator que estimula o uso do \textit{RISC-V} é a modernização dos livros didáticos. A nova versão do livro utilizado em OAC, Organização e Projeto de Computadores, de David Patterson e John Hennessy, utiliza a \textit{ISA RISC-V}.}


\section{O Projeto}

    {A proposta do projeto consiste no desenvolvimento de um processador sintetizável em \textit{FPGA}, utilizando o conjunto de instruções \textit{RISC-V}. O caminho da dados inicialmente implementado será um Uniciclo de 64 bits, já prevendo sua expansão para um \textit{datapath} Multiciclo e um \textit{Pipeline}.}

    {O processador deverá conter os subconjuntos de instruções I (para operações com inteiros, sendo o único subconjunto com implementação mandatória pela arquitetura), M (para multiplicação e divisão de inteiros), F (para ponto flutuante com precisão simples conforme o padrão IEEE 754 com revisão de 2008), possivelmente D (ponto flutuante de precisão dupla) e, uma vez que o caminho de dados terá 64 bits e já precisará de um controle para instruções de tamanho variável, C (para instruções comprimidas em 16 bits). No presente momento, a implementação do subconjunto D é duvidosa e a do subconjunto A (operações atômicas de sincronização) está descartada, e com isso o trabalho desenvolvido não pode ser definido como de propósito geral, G (que deve conter os pacotes I, M, A, F e D). Assim, pela nomenclatura da arquitetura, o processador desenvolvido será um \textit{RV64IMFC}.}

    {O projeto também pretende contemplar \textit{traps}, interrupções, exceções e outras funcionalidades de nível privilegiado da arquitetura. Porém, no presente momento as especificações do nível privilegiado da arquitetura encontram-se em versão \textit{draft}. Com isso, a definição do que será implementado da camada privilegiada deverá aguardar a publicação da versão \textit{standard}.}

    {Uma vez que o projeto utilizará a estrutura do processador \textit{MIPS-PUM} (processador de arquitetura \textit{MIPS32} atualmente desenvolvido na turma de OAC do prof. Marcus Vinicuis Lamar) como referência, as interfaces e controladores dos periféricos serão reaproveitadas, sendo reescritas e melhor documentadas quando possível e/ou necessário.}

    {Como a \textit{ISA RISC-V} é relativamente nova, faltam ferramentas didáticas (como o simulador da \textit{ISA MIPS}, o \textit{MARS}) para uma aplicabilidade adequada do conteúdo em sala de aula e em laboratório. Desta forma, é necessária a implementação de um \textit{IDE} (\textit{Integrated Development Environment}, ou Ambiente de Desenvolvimento Integrado) para a escrita de código \textit{Assembly RISC-V}, além de montagem e simulação do \textit{Assembly}. Uma outra possibilidade é a criação de um pacote com sintaxe, montador e simulador da arquitetura para ser utilizado em um \textit{IDE} já existente (e.g. Atom Editor).}


\end{document}
