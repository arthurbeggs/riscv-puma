\chapter{Introdução}

\label{CapIntro}

% Resumo opcional. Comentar se não usar.
%\resumodocapitulo{Resumo opcional}


\section{Motivação}

    {O mercado de trabalho está a cada dia mais exigente, sempre buscando profissionais que conheçam as melhores e mais recentes ferramentas disponíveis. Além disso, muitos universitários se sentem desestimulados ao estudarem assuntos desatualizados e com baixa possibilidade de aproveitamento do conteúdo no mercado de trabalho. Isso alimenta o desinteresse pelos temas abordados e, em muitos casos, leva à evasão escolar. Assim, é importante renovar as matérias com novas tecnologias e tendências de mercado sempre que possível a fim de instigar o interesse dos discentes e formar profissionais mais capacitados e preparados para as demandas da atualidade.}

    {Hoje, a disciplina OAC (Organização e Arquitetura de Computadores) é ministrada utilizando a arquitetura \textit{MIPS32} (\textit{Microprocessor without Interlocked Pipeline Stages}). Apesar da arquitetura \textit{MIPS32} ainda ter grande força no meio acadêmico (em boa parte devido a sua simplicidade e extensa bibliografia), sua aplicação na indústria tem diminuído consideravelmente na última década.}

    {Embora a curva de aprendizagem de linguagens \textit{Assembly} de alguns processadores \textit{RISC} (\textit{Reduced Instruction Set Computing}) seja relativamente baixa para quem já conhece o \textit{Assembly MIPS32}, aprender uma arquitetura atual traz o benefício de conhecer o \textit{estado da arte} da organização e arquitetura de computadores.}

    {Para a modernização da disciplina, foi escolhida a \textit{ISA} (\textit{Instruction Set Architecture}) \textit{RISC-V} desenvolvida na Divisão de Ciência da Computação da Universidade da Califórnia, Berkeley.}

\section{Por que \textit{RISC-V?}}

    {A \textit{ISA RISC-V} (lê-se \textit{"risk-five"}) é uma arquitetura \textit{open source} com licença \textit{BSD} (\textit{Berkeley Software Distribution}), o que permite o seu livre uso para quaisquer fins, sem distinção de se o trabalho possui código aberto ou fechado. Tal característica possibilita que grandes fabricantes utilizem a arquitetura para criar a base de seus produtos, mas que o desenvolvimento de qualquer subconjunto de instruções não-\textit{standard} ou seus métodos de implementação possam ser fechados, protegendo a propriedade intelectual e estimulando investimentos em pesquisa e desenvolvimento.}

    {Empresas como Google, IBM, AMD, Nvidia, Hewlett Packard, Microsoft, Oracle, Qualcomm e Western Digital são algumas das fundadoras e investidoras da \textit{RISC-V Foundation}, órgão responsável pela governança da arquitetura. Isso demonstra o interesse das gigantes do mercado no sucesso e disseminação da arquitetura.}

    {O conjunto de instruções foi desenvolvido tendo em mente seu uso em diversas escalas: sistemas embarcados, \textit{smartphones}, computadores pessoais, servidores e supercomputadores o que permitirá maior reuso de \textit{software} e maior integração de \textit{hardware}.}

    {Outro fator que estimula o uso do \textit{RISC-V} é a modernização dos livros didáticos. A nova versão do livro utilizado em OAC, Organização e Projeto de Computadores, de David Patterson e John Hennessy, utiliza a \textit{ISA RISC-V}.}
